%\documentclass{beamer}
%
%\title{Python for Journalists}
%\author{PyladiesBCN}
%\date{Mars 19, 2015}
%
%\usetheme{pyladies}
%
%\begin{document}
%\maketitle

\documentclass{beamer}
\usetheme{pyladies}

\usepackage[utf8]{inputenc}
\usepackage[T1]{fontenc}

%% Use any fonts you like.
\usepackage{helvet}

% code highlighting
\usepackage{listings}
\usepackage{color}
 
\definecolor{codegreen}{rgb}{0,0.6,0}
\definecolor{codegray}{rgb}{0.5,0.5,0.5}
\definecolor{codepurple}{rgb}{0.58,0,0.82}
\definecolor{backcolour}{rgb}{0.95,0.95,0.92}
 
\lstdefinestyle{mystyle}{
    backgroundcolor=\color{backcolour},   
    commentstyle=\color{codegreen},
    keywordstyle=\color{magenta},
    numberstyle=\tiny\color{codegray},
    stringstyle=\color{codepurple},
    basicstyle=\scriptsize,
    breakatwhitespace=false,         
    breaklines=true,                 
    captionpos=b,                    
    keepspaces=true,                 
    numbers=left,                    
    numbersep=5pt,                  
    showspaces=false,                
    showstringspaces=false,
    showtabs=false,                  
    tabsize=2
}
 
\lstset{style=mystyle}

\title{\textbf{Python for Journalists}}
\author{\textbf{Cristina Ramón }\\ \textbf{ Núria Pujol }\\ \textbf{ Laura Pérez}}

\date{\today}
\institute{PyladiesBCN}

\begin{document}

\begin{frame}[plain,t]
\titlepage
\end{frame}

\section{PyLadies}
\begin{frame}
\frametitle{Pyladies}
%%Falta omplir una mica de info twitter, meetup etc.
\vspace{1em}
Pyladies is an \textbf{international mentorship group} with a focus on helping more women become active participants and leaders in the Python open-source community.\\
\vspace{1em}
It comprises different groups around the world, including PyladiesBCN! Do you want to learn more about us?\\
\vspace{1em}
	\begin{itemize}
	    \item \textbf{On twitter:} @PyLadiesBCN
		\item \textbf{On meetup:} http://meetup.com/PyLadies-BCN
	\end{itemize}
	
\end{frame}

\section{What’s Python?}
\begin{frame}
\frametitle{What’s Python?}

\begin{center}
{\Large Python is a programming language.}
\end{center}

	\begin{itemize}
	    \item Simple and minimalistic language
		\item Easy to learn compared with other languages.
		\item Free and Open source
	\end{itemize}
\hfill \break
Other more technical features: extensible, embeddable, object oriented, interpreted...
\end{frame}

\section{Some basic concepts}
\begin{frame}
\frametitle{Some basic concepts}

{\normalsize We are going to introduce some basic concepts related to programming in Python. They will be necessary to follow our workshop.\\} 

\vspace{2em}
{\normalsize We assume that you never programmed before.\\}
\vspace{1em}
\textbf{{\Large If you have any questions, please ask!}}

\end{frame}

\subsection{Modules}
\begin{frame}
\frametitle{Modules}

{\normalsize Python is a programming language, that means that we have to write our own functionalities \textbf{but no PANIC!!}\\ Someone have done some of this work for us.\\}
\vspace{1em}
{\normalsize Modules are  pieces of code ready to use, we just need to import them. There are thousands of Python modules (statistics, plotting data, web development, etc.).}

\begin{center}
	\lstinputlisting [linerange={2-2},language=Python]{lines_code.py}
\end{center}

{\normalsize After importing a module we can use all its functions, methods and other functionalities.}

\end{frame}

\subsection{Variables}

\begin{frame}
\frametitle{Variables}
{\normalsize When we write a program we want to reuse some results. For this reason we assign this results (calculus, tables, strings,etc.) to names obtaining variables. \\}
\vspace{1em}
{\normalsize Then, in the following lines of code we can use these variables. \\}
\vspace{1em}
\lstinputlisting [linerange={4-5},language=Python]{lines_code.py}
\vspace{1em}
{\normalsize Notice that if variable names are representative they help us to make our code more readable.}
\end{frame}

\subsection{Lists}
\begin{frame}
\frametitle{Lists}

{\normalsize A list is a collection of variables. Those variables can be words, numbers, other variables, etc. \\}
\vspace{1em}
\lstinputlisting [linerange={7-7},language=Python]{lines_code.py}

\vspace{1em}
{\normalsize To access every item in the list we call it by its position. \textbf{BUT!! we start counting on zero:} \\}
\vspace{1em}
\lstinputlisting [linerange={9-10},language=Python]{lines_code.py}
\end{frame}

\subsection{Methods}
\begin{frame}
\frametitle{Methods}

{\normalsize In Python, every variable has available methods (“functions”). But every variable has different methods depending of variable type (integer, string, etc.). \\}
\vspace{1em}
\lstinputlisting [linerange={12-14},language=Python]{lines_code.py}

{\normalsize For this reason it is said that “all in Python is an object”.\\}

\end{frame}

\subsection{Control Flow}
\begin{frame}
\frametitle{Control Flow Statements}

{\normalsize To give some kind of intelligence to our code and obtain different outputs depending on some conditions we can use control flow statements.   
\\}
\vspace{1em}
\begin{itemize}
	\item \textbf{Conditional} (if, elif, else).
	\item \textbf{Loops} (while , for).
\end{itemize}
\vspace{1em}
{\normalsize Obviously we can mix this two types of control flow statements.}
\end{frame}

\begin{frame}
\frametitle{Control Flow Statements - Examples}
\lstinputlisting [linerange={16-19},language=Python]{lines_code.py}
\vspace{1em}
\lstinputlisting [linerange={21-22},language=Python]{lines_code.py}
\vspace{1em}
{\normalsize You should respect the \textbf{indentation} in Python (those white spaces at the beginning of some lines)}
\end{frame}

\begin{frame}
\frametitle{Control Flow Statements - Examples}
{\normalsize You can also use conditions with ">", "<", "==", "and", "or"\\}
\vspace{1em}
\lstinputlisting [linerange={24-27},language=Python]{lines_code.py}
\end{frame}

\section{Usage}

\begin{frame}

\frametitle{How to use Python?}

{\normalsize To use Python we need to write our code “somewhere”.\\}
\vspace{1em}
{\normalsize Python have 2 different ways of programming:\\}
\vspace{1em}
\begin{itemize}
	\item Using interactive console.
	\item Executing a script.
\end{itemize}

\end{frame}

\section{IPython Notebook}
\begin{frame}
\frametitle{Interactive console. IPython Notebook}

{\normalsize \textbf{IPython Notebook} is a quite new interactive console for programming in Python.\\}
\vspace{1em}
{\normalsize Is not really used by programmers but is widely used in scientific environment.\\}
\vspace{2em}
\textbf{{\normalsize In Interactive Python console we execute our code line by line and we obtain results step by step.\\}} 
\end{frame}

\section{Executing a script}
\begin{frame}
\frametitle{Executing a script}
{\normalsize To execute a script we need a file with .py extension containing our code. Then we  execute this command  in the console:\\}
\vspace{1em}
\lstinputlisting [linerange={29-29},language=bash]{lines_code.py}
\vspace{1em}
{\normalsize We could also use an IDE (Interactive Development Environment) but for beginners is not really necessary.}
\end{frame}

\ThankYouFrame

\end{document}